\documentclass[12pt,a4paper]{article}
\usepackage[utf8]{inputenc}
\usepackage[russian]{babel}
\usepackage[fleqn]{amsmath}
\usepackage[T1]{fontenc}
\usepackage{mathtools}
\usepackage[outline]{contour}
\usepackage{color}
\usepackage{ulem}
\usepackage{indentfirst}
\usepackage{floatflt}

\usepackage{fontspec}
\setmainfont[
	Ligatures=TeX,
	Extension=.otf,
	BoldFont=cmunbx,
	ItalicFont=cmunti,
	BoldItalicFont=cmunbi,
]{cmunrm}
\setsansfont{Arial}
\setmonofont{Consolas}
\RequirePackage{polyglossia}
\setdefaultlanguage{russian}
\setotherlanguage{english}

% Настройка страницы
\textheight=24cm % высота текста
\textwidth=16cm % ширина текста
\oddsidemargin=0pt % отступ от левого края
\topmargin=-1.5cm % отступ от верхнего края
\parindent=24pt % абзацный отступ
\parskip=5pt % интервал между абзацами
\tolerance=2000 % терпимость к "жидким" строкам
\flushbottom % выравнивание высоты страниц

% Объявления от tikz
\usepackage{tikz}
\usepackage{graphics}
\usepackage{xcolor}
\usetikzlibrary{calc}
\usetikzlibrary{through}
\usetikzlibrary{intersections}
\usetikzlibrary{patterns}

% Объявляем новую команду для переноса строки внутри ячейки таблицы
\newcommand{\specialcell}[2][c]{\begin{tabular}[#1]{@{}c@{}}#2\end{tabular}}

% Автор и прочее
\title{Заметки по математике.}
\author{Авторы: Шепрут Илья.}
\date{Алматы, 2016}

\begin{document} \fontsize{12pt}{16pt}\selectfont
%--------------------------------------------------------------------------------%
%--------------------------------------------------------------------------------%

\maketitle

\tableofcontents

\thispagestyle{empty}

\newpage

\setcounter{page}{1} % начать нумерацию с номера 1

%--------------------------------------------------------------------------------%
%--------------------------------------------------------------------------------%

\section{Интегралы}

\subsection{Использование дифференциала}

Как выносить функцию под знак дифференциала, и наоборот, как вносить функцию под дифференциал:
$$ d\, f(x) = f'(x)\, dx $$

$$ f(x)\, dx = d\, F(x) $$

$$ d\, f(x) = d(f(x) + C), C \in R $$

{\bfseries Пример использования: }

$$ \int \tg x\, dx = \int \frac{\sin x}{\cos x}\, dx = \int \frac{-d\, \cos x}{\cos x} = -\ln |\cos x| + C, C \in R $$

\subsection{Интегрирование по частям}

Основная идея: когда под интегралом дробь с линейным уравнением наверху, и с квадртаным снизу, то снизу можно: Либо выделить полный квадрат, а затем заменить переменную, как в примере 1.   Либо разложить линейное уравнение вверху на сумму или разность с разными коэффицентами двух скобок, получаемых из квадратного уравнения снизу. Чтобы потом осталась сумма двух дробей, взять от которых интеграл не составляет труда. Как в примере 2.

{\bfseries Пример 1:}

$$ \int \frac{d\, x}{x^2 + 6\cdot x + 5} = \int \frac{d\, x}{(x^2 + 6\cdot x + 9) + 1} = \int \frac{d\, (x + 3)}{(x + 3)^2 + 1} =  $$
$$ = \left| x+3 = t \right| = \int \frac{d\, t}{t^2 + 1} = \arctg (x + 3) + C, c \in R $$

{\bfseries Пример 2:}

$$ \int \frac{d\, x}{x^2 + 6\cdot x + 5} = \int \frac{d\, x}{(x + 1)\cdot(x + 5)} = \frac{1}{4}\cdot \int \frac{4\cdot d\, x}{(x + 1)\cdot(x + 5)} = \frac{1}{4}\cdot \int \frac{(x + 5) - (x + 1)}{x + 1)\cdot(x + 5)} \cdot d\, x = $$
$$ = \frac{1}{4}\cdot \int \left( \frac{1}{x + 1} - \frac{1}{x + 5} \right)\cdot d\, x = \frac{1}{4}\cdot \ln |x + 1| - \frac{1}{4}\cdot \ln |x + 5| + C, C \in R $$

\subsection{Интегрирование по частям 2}

Эта формула позволяет менять местами подинтегральное и поддифферециальное выражение, что иногда упрощает решение интеграла:
$$ \label{eq:fourierrow} \int f(x) d\, g(x) = f(x)\cdot g(x) - \int g(x) d\, f(x) $$

{\bfseries Пример 1:}
$$ \int x\cdot \sin x\, d\, x = \int x\, d\, \cos x = -x\cdot \cos x + \int \cos x\, d\, x = -x\cdot \cos x + \sin x + C, C \in R $$

\vspace{1cm}

{\bfseries Пример 2:}
$$ \int \overbrace{\strut \arctg x}^{\text{Поместим это}}\, \overbrace{\strut dx}^{\text{сюда}} = x\cdot \arctg x - \int x\, d \overbrace{\strut \arctg x}^{\text{Вытащим обратно}} = x\cdot \arctg x - \int \frac{\overbrace{\strut x}^{\text{Поместим это}}}{x^2 + 1}\, \overbrace{\strut dx}^{\text{сюда}} = $$
$$ = x\cdot \arctg x - \frac{1}{2}\cdot \int \frac{d(x^2 + 1)}{x^2 + 1} = x\cdot \arctg x - \frac{1}{2}\cdot \ln |x^2 + 1| $$

\vspace{1cm}

{\bfseries Пример 3:}

$$ \int \arcsin x\, dx = x\cdot \arcsin x - \int x\,d \arcsin x = x\cdot \arcsin x - \int \frac{x}{\sqrt{\mathstrut 1 - x^2}}\,dx = $$ $$ = x\cdot \arcsin x + \frac{1}{2}\cdot \int \frac{d(1-x^2)}{\sqrt{\mathstrut 1-x^2}} = x\cdot \arcsin x + \frac{1}{2}\cdot \int \frac{dt}{\sqrt{\mathstrut t}} = x\cdot \arcsin x + \sqrt{\mathstrut 1-x^2} $$

\vspace{1cm}

{\bfseries Решение сложного интеграла по действиям:}
$$ 1)\,\,\int \overbrace{\strut \arctg \sqrt{\mathstrut x}}^{\text{Поместим это}}\, \overbrace{\strut dx}^{\text{сюда}} = x\cdot \arctg \sqrt{\mathstrut x} - \int x\, d\overbrace{\strut \arctg \sqrt{\mathstrut x}}^{\text{Вытащим обратно}} = x\cdot \arctg \sqrt{\mathstrut x} - \int \frac{x}{(x + 1)\cdot 2\cdot \sqrt{\mathstrut x}}\, dx = $$ $$ = x\cdot \arctg \sqrt{\mathstrut x} - \frac{1}{2}\cdot \underbrace{\strut \int \frac{\sqrt{\mathstrut x}}{x + 1}\, dx}_{\text{Надо решить это}}; $$

$$ 2)\,\,\int \frac{\sqrt{\mathstrut x}}{x + 1}\, dx = |t = \sqrt{\mathstrut x}| = \int \frac{t}{t^2 + 1}\,  \overbrace{\strut dt^2}^{\text{Вытащим это}} = 2\cdot \int \frac{t^2+1-1}{t^2+1} = 2\cdot \int \frac{t^2+1}{t^2+1}\,dt-2\cdot\int\frac{dt}{t^2+1} = $$ $$ = 2\cdot t-2\arctg t = 2\cdot\sqrt{\mathstrut x}-2\cdot\arctg\sqrt{\mathstrut x}+C,C\in R; $$

$$ 3)\,\,x\cdot \arctg \sqrt{\mathstrut x} - \sqrt{\mathstrut x}+\arctg\sqrt{\mathstrut x}+C = (x+1)\cdot\arctg \sqrt{\mathstrut x}-\sqrt{\mathstrut x}+C,C\in R $$

\subsection{Площадь криволинейной трапеции}

\begin{center}
\noindent
\begin{tikzpicture}
	% Создание собственной функции
	[ declare function={ myfun(\x) = sin(\x*360/pi + 30)/4 + 1.7; }, ]

	% Присвоение значения размера графика
	\def\strtx{-0.2}
	\def\strty{-0.2}

	\def\endx{3.5}
	\def\endy{2.6}

	% Функция графика
	\def\normaltwo{\x,{myfun(\x)}}

	% Описание положения и рассчет координат для горизонтали
	\def\b{(\endx - \strtx)*0.8 + \strtx}
	\def\fb{myfun(\b)}
	\def\a{(\endx - \strtx)*0.25 + \strtx}
	\def\fa{myfun(\a)}

	\def\xe{(\endx - \strtx)*0.6 + \strtx}
	\def\fxe{myfun(\xe)}
	\def\xs{(\endx - \strtx)*0.4 + \strtx}
	\def\fxs{myfun(\xs)}

	%--------------------------------------------------------------------------------%

	% Оси, сетка
	\draw[->] (\strtx,0) -- (\endx,0) node[right] {x};
	\draw[->] (0,\strty) -- (0,\endy) node[above] {y};
	\draw[step=.4cm,gray,very thin] (\strtx+0.1,\strty+0.1) grid (\endx-0.1,\endy-0.1);

	% Штриховка площади
	\fill [pattern=north east lines, pattern color=gray!60] ({\a},0) -- plot[domain=\a:\b] (\normaltwo) -- ({\b},0) -- cycle;
	\fill [pattern=north west lines, pattern color=gray!60] ({\xs},0) -- plot[domain=\xs:\xe] (\normaltwo) -- ({\xe},0) -- cycle;

	% Сам график
	\draw[color=blue,domain=\strtx:\endx] plot (\normaltwo) node[right] {$y = f(x)$};

	% Рисование площади обычной
	\draw[color=red, very thick] (0,0) -- plot[domain=0:\a] (\normaltwo) -- ({\a},0) -- cycle;

	% Обозначение и линия горизонтали
	\draw[dashed] ({\b},{\fb}) -- ({\b},0) node[below] {$b$};
	\draw[dashed] ({\a},{\fa}) -- ({\a},0) node[below] {$a$};
	\draw[dashed] ({\xe},{\fxe}) -- ({\xe},0) node[below] {$\scriptstyle x_0 + \Delta x$};
	\draw[dashed] ({\xs},{\fxs}) -- ({\xs},0) node[below] {$\scriptstyle x_0$};

	\draw (1.65, 1) node[anchor=north] {$\scriptstyle \Delta S$};
	\draw (0.35, 1) node[color=red,anchor=north] {$\scriptstyle S(a)$};
\end{tikzpicture}
\end{center}

{\itshape Определение}: криволинейной трапецией называет множество точек плоскости, ограниченные осью ОХ, прямыми $ x = a $, $ x = b $ и графиком функции $ y = f(x) $, где $ f(x)>0 $. На рисунке эта площадь показана серой штриховкой. Вывод площади криволинейной трапеции:

Обозначим $ S(x) $ как площадь криволинейной трапеции от горизонтали $ 0 $ до горизонтали $ x $, сверху ограниченой графиком функции $ y = f(x) $, а снизу ограниченной вертикалью $ 0 $. На рисунке показаны границы считаемой площади для $ S(a) $.

$$ \text{Тогда обозначим\ } \Delta S = S(x_0+\Delta x)-S(x_0) $$

$$ \text{А так же справедливо\ } \Delta S = \Delta x\cdot f(x_0)\text{, при\,} \Delta x \rightarrow 0 $$

\begin{center}
	Тогда приравняем их и поставим пределы:
\end{center}

$$ \lim_{\Delta x \to 0} (S(x_0+\Delta x)-S(x_0)) = \lim_{\Delta x \to 0} (\Delta x\cdot f(x)) $$

$$ \lim_{\Delta x \to 0} \frac{S(x_0+\Delta x)-S(x_0)}{\Delta x} = \lim_{\Delta x \to 0} f(x) $$

$$ S'(x_0) = f(x_0) \text{\ --- по определению производной, и т. к. $ x_0 $ произвольная.} $$

$$ \text{Следовательно\ } S(x) = \int f(x)\, dx = F(x) + C,\ C\in R $$

\begin{center}
	Тогда формула площади нужной части будет:
\end{center}

$$ S = S(b) - S(a) = F(b) + C - F(a) - C = F(b) - F(a) $$

\begin{center}
	И специальное обозначение этой площади:
\end{center}

$$ \boxed{S = \int\limits_a^b f(x)\, dx = F(b) - F(a)} $$

\subsection{Геометрическое приложение определенного интеграла}

\subsubsection{Площадь круга}

\begin{center}
\noindent
\begin{tikzpicture}
	% Создание собственной функции
	\tikzset{declare function={	my1(\x) = sqrt(4-(\x)^2);}}

	% Присвоение значения размера графика
	\def\strtx{-1.2}
	\def\strty{-0.2}

	\def\endx{3.0}
	\def\endy{3.0}

	% Функция графика
	\def\normaltwo{\x,{my1(\x)}}
	\def\normalthree{\x,{my2(\x)}}

	% Описание положения и рассчет координат для горизонтали
	\def\b{(\endx - \strtx)*0.8 + \strtx}
	\def\fb{my1(\b)}
	\def\a{(\endx - \strtx)*0.25 + \strtx}
	\def\fa{my1(\a)}

	\def\xe{(\endx - \strtx)*0.55 + \strtx}
	\def\fxe{my1(\xe)}
	\def\xs{(\endx - \strtx)*0.5 + \strtx}
	\def\fxs{my1(\xs)}

	%--------------------------------------------------------------------------------%

	% Оси, сетка
	\draw[->] (\strtx,0) -- (\endx,0) node[right] {x};
	\draw[->] (0,\strty) -- (0,\endy) node[above] {y};
	\draw[step=.4cm,gray,very thin] (\strtx+0.1,\strty+0.1) grid (\endx-0.1,\endy-0.1);

	% Штриховка площади
	\fill [pattern=north east lines, pattern color=gray!60] (0,0) -- plot[domain=0:2, samples=100,smooth] (\normaltwo) -- (2,0) -- cycle;

	% Сам график
	\draw[color=blue,domain={\strtx}:2] plot[samples=100,smooth] (\normaltwo);

	\draw[color=blue] (1.5,2.4) node {$y = \sqrt{\mathstrut R^2 - x^2}$};

	% Обозначение и линия горизонтали
	\draw (1, 0.5) node {$\scriptstyle S_{\frac{1}{4}}$};

	\draw (2,-2pt) -- (2,2pt) node[anchor=north] {$R$};
	\draw (-2pt,2) -- (2pt,2) node[anchor=east] {$R$};
\end{tikzpicture}
\end{center}

Формула полукруга: $ y = \sqrt{\mathstrut R^2 - x^2}$. Тогда рассчитаем его площадь:

$$ S = \int\limits_0^R \sqrt{\mathstrut R^2 - x^2}\, dx = \int\limits_0^R R\cdot \sqrt{\mathstrut 1 - \left( \frac{x}{R}\right)^2}\, dx = \left| \frac{x}{R} = \cos t;\ x = R\cdot \cos t;\ a = \frac{\pi}{2};\ b = 0 \right| = $$  $$ \int\limits_{\frac{\pi}{2}}^0 R\cdot \sqrt{\mathstrut 1 - \cos t} \, d R\cdot \cos t = R^2\cdot \int\limits_{\frac{\pi}{2}}^0 -\sin^2 t\, dt = R^2\cdot \int\limits_{\frac{\pi}{2}}^0 \frac{\cos 2\cdot t - 1}{2}\, dt = R^2\cdot \left.\left(\frac{1}{4}\cdot \sin 2\cdot t - \frac{1}{2}\cdot t\right)\right|_{\frac{\pi}{2}}^0 = $$ $$ = R^2\cdot (0 - 0 - \left(0 - \frac{\pi}{4} \right)) = \frac{\pi\cdot R^2}{4} $$

$$ S_{\text{круга}} = 4\cdot \frac{\pi\cdot R^2}{4} = \pi\cdot R^2 $$

\subsubsection{Объем тела вращения}

\newcommand{\Ellipse}[4][]
{
	\draw[#1] ({#3},-{#2(#3)}) arc [start angle=-90, end angle=90, x radius={#4}, y radius={#2(#3)}];
	\draw[dashed, #1] ({#3},{#2(#3)}) arc [start angle=90, end angle=270, x radius={#4}, y radius={#2(#3)}];
}

\begin{center}
\noindent
\begin{tikzpicture}
	% Создание собственной функции
	\tikzset{declare function={	my1(\x) = ((\x-4)^2)/16 + 1.3;
					my2(\x) = -my1(\x);}}

	% Присвоение значения размера графика
	\def\strtx{-0.2}
	\def\strty{-2.6}

	\def\endx{5.5}
	\def\endy{2.6}

	% Функция графика
	\def\normaltwo{\x,{my1(\x)}}
	\def\normalthree{\x,{my2(\x)}}

	% Описание положения и рассчет координат для горизонтали
	\def\b{(\endx - \strtx)*0.8 + \strtx}
	\def\fb{my1(\b)}
	\def\a{(\endx - \strtx)*0.25 + \strtx}
	\def\fa{my1(\a)}

	\def\xe{(\endx - \strtx)*0.55 + \strtx}
	\def\fxe{my1(\xe)}
	\def\xs{(\endx - \strtx)*0.5 + \strtx}
	\def\fxs{my1(\xs)}

	%--------------------------------------------------------------------------------%

	% Оси, сетка
	\draw[->] (\strtx,0) -- (\endx,0) node[right] {x};
	\draw[->] (0,\strty) -- (0,\endy) node[above] {y};
	\draw[step=.4cm,gray,very thin] (\strtx+0.1,\strty+0.1) grid (\endx-0.1,\endy-0.1);

	% Штриховка площади
	\fill [pattern=north east lines, pattern color=gray!60] ({\a},0) -- plot[domain=\a:\b] (\normaltwo) -- ({\b},0) -- cycle;
	\fill [fill=gray, fill opacity=0.5] ({\xs},0) -- plot[domain=\xs:\xe] (\normaltwo) -- ({\xe},0) -- cycle;

	% Сам график
	\draw[color=blue,domain=\strtx:\endx] plot (\normaltwo) node[right] {$y = f(x)$};
	\draw[color=blue,domain=\strtx:\endx] plot (\normalthree) node[right] {};

	\Ellipse{my1}{\a}{0.5};
	\Ellipse{my1}{\b}{0.5};

	\Ellipse{my1}{\xs}{0.5};
	\Ellipse{my1}{\xe}{0.5};

	% Обозначение и линия горизонтали
	\draw ({\b},{\fb}) -- ({\b},0) node[below] {$b$};
	\draw ({\a},{\fa}) -- ({\a},0) node[below] {$a$};
	\draw ({\xe},{\fxe}) -- ({\xe},0) node[right, pos=0.88] {\contour{white}{$\scriptstyle x_0 + \Delta x$}};
	\draw ({\xs},{\fxs}) -- ({\xs},0) node[left, pos=0.9] {\contour{white}{$\scriptstyle x_0$}};

	\draw ({\xs}, 0) node[anchor=north] {$\scriptstyle \Delta V$};
\end{tikzpicture}
\end{center}

Вывод формулы:

$$ \Delta V = V(x_0 + \Delta x) - V(x_0) $$

$$ \Delta X = \pi\cdot f^2(x_0)\cdot \Delta x \text{,\ при\ } \Delta x \rightarrow 0 $$

Приравняем при $ \Delta x \rightarrow 0 $, поделив на $ \Delta x $:

$$ \lim_{\Delta x \to 0} \cfrac{V(x_0 + \Delta x) - V(x_0)}{\Delta x} = \lim_{\Delta x \to 0} (\pi\cdot f^2(x_0)\cdot) $$

$$ V'(x_0) = \pi\cdot f^2(x_0) \text{ --- по определению производной. Следовательно:} $$

$$ \boxed{V = \int\limits_a^b \pi\cdot f^2(x)\, dx} $$

\subsubsection{Объем цилиндра}

$$ V = \pi \cdot \int\limits_0^H R^2\, dx = \pi\cdot R^2\cdot H $$

\subsubsection{Объем конуса}

Нахождение формулы прямой секущей конуса:

$$ y = k\cdot x; k = \tg \alpha = \frac{R}{H} $$

Вывод объема:

$$ V = \pi\cdot \int\limits_0^H \frac{R^2}{H^2}\cdot x^2\, dx = \frac{\pi\cdot R^2}{H^2}\cdot \left. \frac{x^3}{3}\right|_0^H = \frac{\pi}{3}\cdot R^2\cdot H $$

\subsubsection{Объем усеченного конуса}

Формула прямой секущей конуса:

$$ y = \frac{R_2-R_1}{H}\cdot x + R_1 $$

Вывод формулы объема:

$$ V = \pi \int\limits_0^H \left( \frac{(R_2-R_1)^2}{H^2}\cdot x^2 + 2\cdot \frac{R_2-R_1}{H}\cdot x\cdot R_1 + R_1^2 \right)\, dx = $$ $$ = \pi \left. \left( \frac{(R_2-R_1)^2}{H^2}\cdot \frac{x^3}{3} + 2\cdot \frac{R_2-R_1}{H}\cdot \frac{x^2}{2}\cdot R_1 + R_1^2\cdot x \right)\right|_0^H = $$ $$ = \frac{\pi\cdot R}{3}(R_2^2 - 2\cdot R_1\cdot R_2 + R_1^2 + 3\cdot R_2\cdot R_1 - 3\cdot R_1^2 + 3\cdot R_1^2) = \boxed{\frac{\pi\cdot H}{3}(R_1^2 + R_1\cdot R_2 + R_2^2)} $$

\subsubsection{Объем тела через S сечения}

\newcommand{\EllipseH}[4][]
{
	\draw[#1] (#3,{#2(#3))}) arc [start angle=-180, end angle=0, y radius={#4}, x radius={2-#3}];
	\draw[dashed, #1] (2 + 2-#3,{#2(#3)}) arc [start angle=0, end angle=180, y radius={#4}, x radius={2-#3}];
}

\begin{center}
\noindent
\begin{tikzpicture}
	% Создание собственной функции
	\tikzset{declare function={	my1(\x) = -((\x-2)^2) + 2.3;
					my2(\x) = -my1(\x);}}

	% Присвоение значения размера графика
	\def\strtx{-0.2}
	\def\strty{-2.6}

	\def\endx{5.5}
	\def\endy{2.6}

	% Функция графика
	\def\normaltwo{\x,{my1(\x)}}
	\def\normalthree{\x,{my2(\x)}}

	% Описание положения и рассчет координат для горизонтали
	\def\b{(\endx - \strtx)*0.8 + \strtx}
	\def\fb{my1(\b)}
	\def\a{(\endx - \strtx)*0.25 + \strtx}
	\def\fa{my1(\a)}

	\def\xe{(\endx - \strtx)*0.55 + \strtx}
	\def\fxe{my1(\xe)}
	\def\xs{(\endx - \strtx)*0.5 + \strtx}
	\def\fxs{my1(\xs)}

	%--------------------------------------------------------------------------------%

	% Оси, сетка
	\draw[->] (0,0) -- (\endx,0) node[right] {$x$};
	\draw[->] (0,0) -- (0,\endy) node[above] {$y$};
	\draw[->] (0,0) -- (-1.5,-2.25) node[above] {$z$};

	% Сам график
	\draw[color=blue,domain=0.1:3.9] plot (\normaltwo);
	\EllipseH{my1}{0.1}{0.5};
	\EllipseH{my1}{0.6}{0.5};
	\EllipseH{my1}{0.7}{0.5};

	% Линии обозначающие размер
	\draw (2,{my1(2)}) -- (5,{my1(2)});
	\draw (2,{my1(3.9)}) -- (5,{my1(3.9)});
	\draw[<->] (4.7,{my1(2)-0.02}) -- (4.7,{my1(3.9)+0.02}) node[right, pos=0.5] {$\scriptstyle H$};

	\draw (2+2-0.6,{my1(2+2-0.6)}) -- (4,{my1(2+2-0.6)});
	\draw (2+2-0.7,{my1(2+2-0.7)}) -- (4,{my1(2+2-0.7)});
	\draw[<->] (3.9,{my1(2+2-0.6)+0.02}) -- (3.9,{my1(2+2-0.7)-0.02}) node[right, pos=0.5] {$\scriptstyle \Delta h$};

	\draw[<->] (3.9,{my1(2+2-0.6)-0.02}) -- (3.9,{my1(2+2-0.1)+0.02}) node[right, pos=0.5] {$\scriptstyle h$};

	% Подпись объема
	\draw (2,{my1(0.65)}) node {$\scriptstyle \Delta V$};
\end{tikzpicture}
\end{center}

$$ \Delta V = V(h + \Delta h) - V(h) $$

$$ \Delta V = S(h)\cdot \Delta h $$

$$ \lim_{\Delta h \to 0} \frac{V(h + \Delta h) - V(h)}{\Delta h} = \lim_{\Delta h \to 0} S(h) $$

$$ V'(h) = S(h) $$

$$ \boxed{V = \int\limits_0^S S(h)\, dh} $$

%--------------------------------------------------------------------------------%
%--------------------------------------------------------------------------------%
\end{document}